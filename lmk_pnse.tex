\documentclass[a4paper]{llncs}

\usepackage[utf8]{inputenc}
\usepackage[T1]{fontenc}
\usepackage{url}
\usepackage{graphicx}
\usepackage{subfigure}
\usepackage{listings}
\usepackage{authblk}
%\usepackage{amsthm}
\usepackage{amssymb}

\usepackage{tabularx}
\usepackage{verbatim}
\usepackage{wrapfig}
\usepackage{capt-of}

\newcommand{\keep}[1]{}

\newcommand{\smlcode}[1]{{\texttt{#1}}}
\newcommand{\figitem}[1]{\textsf{#1}}
\newcommand{\pragmatic}[1]{$\langle$$\langle$\figitem{#1}$\rangle$$\rangle$} 
\newcommand{\concept}[1]{{\emph{#1}}}

 \newcommand{\com}[1]{
         \mbox{}
        \marginpar{\hrule\footnotesize\raggedright\hspace{0pt}#1\vspace{2mm}}
 }

\newcommand{\oldcom}[1]{{}}

\newcommand{\okcom}[1]{{}}

\newcommand{\postponedcom}[1]{{}}

\newcommand{\ignore}[1]{{}}

\newcommand{\original}[1]{{}}

\newcommand{\reduce}[1]{{}}

%\renewcommand{\com}[1]{\footnote{#1}}

%\renewcommand{\com}[1]{}

% \theoremstyle{definition}
% \newtheorem{definition}{Definition}
% \newtheorem{proposition}[definition]{Proposition}

\keep{
\lstset{%language=groovy,
   basicstyle=\ttfamily\small %
%  commentstyle=\itshape,%
%  stringstyle=\itshape,keywordstyle=\bf,tabsize=2,numberstyle=\tiny,%
%  stepnumber=5,firstnumber=1,numberfirstline=true,%
%  showstringspaces=false,numbers=left,%
%  numberbychapter=false,belowcaptionskip=\smallskipamount,basewidth=.5em,
%  escapechar=\$
}
}



% eki: suggest to switch to the following use of listings for the final version
%
\renewcommand{\ttdefault}{pcr}  

\lstset{
  language=Java,
  basicstyle=\scriptsize\ttfamily,
  keywordstyle=\bfseries,
  showstringspaces=false,
%   alsoletter={\%},               % Just in case we want %%yield%%
%   morekeywords={\%\%yield\%\%},  % to be shown as keyword (but does not look
%                                  % too nice for me)
  moredelim=**[is][\color{blue}]{@}{@}
}

  
%\title{Implementing the WebSocket Protocol \\ based on Coloured Petri Nets Modelling and \\ Automated Code Generation}
 
\title{An Approach for the Engineering of Protocol Software from Coloured Petri Net Models: \\ A Case Study of the IETF WebSocket Protocol} 
\author{Lars Michael Kristensen}
\institute{Department of Computing, Bergen University College, Norway
\\
Email: \email{lmkr@hib.no} 
}

\pagestyle{empty}
\begin{document}
  
\maketitle

\begin{center}
{\textbf{Invited Talk}}
\end{center}

% The relevance of formal modelling of protocols 
The vast majority of software systems today can be characterised as
concurrent and distributed systems as their operation inherently
relies on protocols executed between independently scheduled software
components. The engineering of correct protocols can
be a challenging task due to their complex behaviour which may result
in subtle errors if not carefully designed. Ensuring
interoperability between independently made implementations is also
challenging due to ambiguous protocol specifications.  Model-based
software engineering offers several attractive benefits for the
implementation of protocols, including automated code generation for
different platforms from design-level models. Furthermore, the use of
formal modelling in combination with model checking
provides techniques to support the development of reliable protocol
implementations.

% CPNs and protocol modelling and verification
Coloured Petri Nets (CPNs) \cite{CPNTOOLSPAPER} is formal language
combining Petri Nets with a programming language to obtain a modelling
language that scales to large systems. In CPNs, Petri Nets provide the
primitives for modelling concurrency and synchronisation while the
Standard ML programming language provides the primitives for modelling
data and data manipulation. CPNs have been successfully applied for
the modelling and verification of many protocols, including Internet
protocols such as the TCP, DCCP, and DYMO protocols
\cite{billington04,acpn2012}. Formal modelling and verification have
been useful in gaining insight into the operation of the protocols
considered and have resulted in improved protocol
specifications. However, earlier work has not fully leveraged the
investment in modelling by also taking the step to automated code
generation as a way to obtain an implementation of the protocol under
consideration.

% PetriCode approach 

In earlier work~\cite{SKK13-2}, we have proposed the PetriCode
approach and developed a supporting software
tool~\cite{petricodePaper} for automatically generating protocol
implementations based on CPN models.  The basic idea of the approach
is to enforce particular modelling patterns and annotate the CPN
models with code generation \concept{pragmatics}. The pragmatics are
bound to code generation templates and used to direct a template-based
model-to-text transformation that generates the protocol
implementation. As part of earlier work, we have demonstrated the use
of the PetriCode approach on small protocols. In addition, we have
shown that our approach supports code generation for multiple
platforms, and that it leads to code that is readable and also %upwards
%and downwards 
 compatible with other software~\cite{ecmfa14}.

% the web socket protocol and contributions of this work

In the present work we consider the application of our code generation 
approach as
implemented in the PetriCode tool to obtain protocol software
implementing the IETF WebSocket protocol~\cite{fette2011websocket}
protocol for the Groovy language and platform. This demonstrates that
our approach and tool scales to industrial-sized protocols. The
WebSocket protocol is a relatively new protocol %currently under
developed by the IETF. The WebSocket protocol makes it possible to
upgrade an HTTP connection to an efficient message-based full-duplex
connection.  WebSocket has already become a popular protocol for
several web-based applications such as games and media streaming services
where bi-directional communication with low latency is needed.

The contributions of our work include showing how we have been able to
model the WebSocket protocol following the PetriCode modelling
conventions. Furthermore, we perform formal verification of the CPN
model prior to code generation, and test the implementation for
interoperability against the Autobahn WebSocket test-suite
\cite{Autobahn} resulting in 97\% and 99\% success rate for the client
and server implementation, respectively. The tests show that the cause
of test failures were mostly due to local and trivial errors in newly
written code-generation templates, and not related to the overall
logical operation of the protocol as specified by the CPN
model. Finally, we demonstrate in this paper that the generated code
is interoperable with other WebSocket implementations.
  
\paragraph{Acknowledgement.} The results presented in this invited talk is based on joint work with Kent I.F. Simonsen, Bergen University College and the Technical University of Denmark, and Ekkart Kindler, the Technical
University of Denmark.

\bibliographystyle{plain}

%\bibliography{bib}

\documentclass[a4paper]{llncs}

\usepackage[utf8]{inputenc}
\usepackage[T1]{fontenc}
\usepackage{url}
\usepackage{graphicx}
\usepackage{subfigure}
\usepackage{listings}
\usepackage{authblk}
%\usepackage{amsthm}
\usepackage{amssymb}

\usepackage{tabularx}
\usepackage{verbatim}
\usepackage{wrapfig}
\usepackage{capt-of}

\newcommand{\keep}[1]{}

\newcommand{\smlcode}[1]{{\texttt{#1}}}
\newcommand{\figitem}[1]{\textsf{#1}}
\newcommand{\pragmatic}[1]{$\langle$$\langle$\figitem{#1}$\rangle$$\rangle$} 
\newcommand{\concept}[1]{{\emph{#1}}}

 \newcommand{\com}[1]{
         \mbox{}
        \marginpar{\hrule\footnotesize\raggedright\hspace{0pt}#1\vspace{2mm}}
 }

\newcommand{\oldcom}[1]{{}}

\newcommand{\okcom}[1]{{}}

\newcommand{\postponedcom}[1]{{}}

\newcommand{\ignore}[1]{{}}

\newcommand{\original}[1]{{}}

\newcommand{\reduce}[1]{{}}

%\renewcommand{\com}[1]{\footnote{#1}}

%\renewcommand{\com}[1]{}

% \theoremstyle{definition}
% \newtheorem{definition}{Definition}
% \newtheorem{proposition}[definition]{Proposition}

\keep{
\lstset{%language=groovy,
   basicstyle=\ttfamily\small %
%  commentstyle=\itshape,%
%  stringstyle=\itshape,keywordstyle=\bf,tabsize=2,numberstyle=\tiny,%
%  stepnumber=5,firstnumber=1,numberfirstline=true,%
%  showstringspaces=false,numbers=left,%
%  numberbychapter=false,belowcaptionskip=\smallskipamount,basewidth=.5em,
%  escapechar=\$
}
}



% eki: suggest to switch to the following use of listings for the final version
%
\renewcommand{\ttdefault}{pcr}  

\lstset{
  language=Java,
  basicstyle=\scriptsize\ttfamily,
  keywordstyle=\bfseries,
  showstringspaces=false,
%   alsoletter={\%},               % Just in case we want %%yield%%
%   morekeywords={\%\%yield\%\%},  % to be shown as keyword (but does not look
%                                  % too nice for me)
  moredelim=**[is][\color{blue}]{@}{@}
}

  
%\title{Implementing the WebSocket Protocol \\ based on Coloured Petri Nets Modelling and \\ Automated Code Generation}
 
\title{An Approach for the Engineering of Protocol Software from Coloured Petri Net Models: \\ A Case Study of the IETF WebSocket Protocol} 
\author{Lars Michael Kristensen}
\institute{Department of Computing, Bergen University College, Norway
\\
Email: \email{lmkr@hib.no} 
}

\pagestyle{empty}
\begin{document}
  
\maketitle

\begin{center}
{\textbf{Invited Talk}}
\end{center}

% The relevance of formal modelling of protocols 
The vast majority of software systems today can be characterised as
concurrent and distributed systems as their operation inherently
relies on protocols executed between independently scheduled software
components. The engineering of correct protocols can
be a challenging task due to their complex behaviour which may result
in subtle errors if not carefully designed. Ensuring
interoperability between independently made implementations is also
challenging due to ambiguous protocol specifications.  Model-based
software engineering offers several attractive benefits for the
implementation of protocols, including automated code generation for
different platforms from design-level models. Furthermore, the use of
formal modelling in combination with model checking
provides techniques to support the development of reliable protocol
implementations.

% CPNs and protocol modelling and verification
Coloured Petri Nets (CPNs) \cite{CPNTOOLSPAPER} is formal language
combining Petri Nets with a programming language to obtain a modelling
language that scales to large systems. In CPNs, Petri Nets provide the
primitives for modelling concurrency and synchronisation while the
Standard ML programming language provides the primitives for modelling
data and data manipulation. CPNs have been successfully applied for
the modelling and verification of many protocols, including Internet
protocols such as the TCP, DCCP, and DYMO protocols
\cite{billington04,acpn2012}. Formal modelling and verification have
been useful in gaining insight into the operation of the protocols
considered and have resulted in improved protocol
specifications. However, earlier work has not fully leveraged the
investment in modelling by also taking the step to automated code
generation as a way to obtain an implementation of the protocol under
consideration.

% PetriCode approach 

In earlier work~\cite{SKK13-2}, we have proposed the PetriCode
approach and developed a supporting software
tool~\cite{petricodePaper} for automatically generating protocol
implementations based on CPN models.  The basic idea of the approach
is to enforce particular modelling patterns and annotate the CPN
models with code generation \concept{pragmatics}. The pragmatics are
bound to code generation templates and used to direct a template-based
model-to-text transformation that generates the protocol
implementation. As part of earlier work, we have demonstrated the use
of the PetriCode approach on small protocols. In addition, we have
shown that our approach supports code generation for multiple
platforms, and that it leads to code that is readable and also %upwards
%and downwards 
 compatible with other software~\cite{ecmfa14}.

% the web socket protocol and contributions of this work

In the present work we consider the application of our code generation 
approach as
implemented in the PetriCode tool to obtain protocol software
implementing the IETF WebSocket protocol~\cite{fette2011websocket}
protocol for the Groovy language and platform. This demonstrates that
our approach and tool scales to industrial-sized protocols. The
WebSocket protocol is a relatively new protocol %currently under
developed by the IETF. The WebSocket protocol makes it possible to
upgrade an HTTP connection to an efficient message-based full-duplex
connection.  WebSocket has already become a popular protocol for
several web-based applications such as games and media streaming services
where bi-directional communication with low latency is needed.

The contributions of our work include showing how we have been able to
model the WebSocket protocol following the PetriCode modelling
conventions. Furthermore, we perform formal verification of the CPN
model prior to code generation, and test the implementation for
interoperability against the Autobahn WebSocket test-suite
\cite{Autobahn} resulting in 97\% and 99\% success rate for the client
and server implementation, respectively. The tests show that the cause
of test failures were mostly due to local and trivial errors in newly
written code-generation templates, and not related to the overall
logical operation of the protocol as specified by the CPN
model. Finally, we demonstrate in this paper that the generated code
is interoperable with other WebSocket implementations.
  
\paragraph{Acknowledgement.} The results presented in this invited talk is based on joint work with Kent I.F. Simonsen, Bergen University College and the Technical University of Denmark, and Ekkart Kindler, the Technical
University of Denmark.

\bibliographystyle{plain}

%\bibliography{bib}

\documentclass[a4paper]{llncs}

\usepackage[utf8]{inputenc}
\usepackage[T1]{fontenc}
\usepackage{url}
\usepackage{graphicx}
\usepackage{subfigure}
\usepackage{listings}
\usepackage{authblk}
%\usepackage{amsthm}
\usepackage{amssymb}

\usepackage{tabularx}
\usepackage{verbatim}
\usepackage{wrapfig}
\usepackage{capt-of}

\newcommand{\keep}[1]{}

\newcommand{\smlcode}[1]{{\texttt{#1}}}
\newcommand{\figitem}[1]{\textsf{#1}}
\newcommand{\pragmatic}[1]{$\langle$$\langle$\figitem{#1}$\rangle$$\rangle$} 
\newcommand{\concept}[1]{{\emph{#1}}}

 \newcommand{\com}[1]{
         \mbox{}
        \marginpar{\hrule\footnotesize\raggedright\hspace{0pt}#1\vspace{2mm}}
 }

\newcommand{\oldcom}[1]{{}}

\newcommand{\okcom}[1]{{}}

\newcommand{\postponedcom}[1]{{}}

\newcommand{\ignore}[1]{{}}

\newcommand{\original}[1]{{}}

\newcommand{\reduce}[1]{{}}

%\renewcommand{\com}[1]{\footnote{#1}}

%\renewcommand{\com}[1]{}

% \theoremstyle{definition}
% \newtheorem{definition}{Definition}
% \newtheorem{proposition}[definition]{Proposition}

\keep{
\lstset{%language=groovy,
   basicstyle=\ttfamily\small %
%  commentstyle=\itshape,%
%  stringstyle=\itshape,keywordstyle=\bf,tabsize=2,numberstyle=\tiny,%
%  stepnumber=5,firstnumber=1,numberfirstline=true,%
%  showstringspaces=false,numbers=left,%
%  numberbychapter=false,belowcaptionskip=\smallskipamount,basewidth=.5em,
%  escapechar=\$
}
}



% eki: suggest to switch to the following use of listings for the final version
%
\renewcommand{\ttdefault}{pcr}  

\lstset{
  language=Java,
  basicstyle=\scriptsize\ttfamily,
  keywordstyle=\bfseries,
  showstringspaces=false,
%   alsoletter={\%},               % Just in case we want %%yield%%
%   morekeywords={\%\%yield\%\%},  % to be shown as keyword (but does not look
%                                  % too nice for me)
  moredelim=**[is][\color{blue}]{@}{@}
}

  
%\title{Implementing the WebSocket Protocol \\ based on Coloured Petri Nets Modelling and \\ Automated Code Generation}
 
\title{An Approach for the Engineering of Protocol Software from Coloured Petri Net Models: \\ A Case Study of the IETF WebSocket Protocol} 
\author{Lars Michael Kristensen}
\institute{Department of Computing, Bergen University College, Norway
\\
Email: \email{lmkr@hib.no} 
}

\pagestyle{empty}
\begin{document}
  
\maketitle

\begin{center}
{\textbf{Invited Talk}}
\end{center}

% The relevance of formal modelling of protocols 
The vast majority of software systems today can be characterised as
concurrent and distributed systems as their operation inherently
relies on protocols executed between independently scheduled software
components. The engineering of correct protocols can
be a challenging task due to their complex behaviour which may result
in subtle errors if not carefully designed. Ensuring
interoperability between independently made implementations is also
challenging due to ambiguous protocol specifications.  Model-based
software engineering offers several attractive benefits for the
implementation of protocols, including automated code generation for
different platforms from design-level models. Furthermore, the use of
formal modelling in combination with model checking
provides techniques to support the development of reliable protocol
implementations.

% CPNs and protocol modelling and verification
Coloured Petri Nets (CPNs) \cite{CPNTOOLSPAPER} is formal language
combining Petri Nets with a programming language to obtain a modelling
language that scales to large systems. In CPNs, Petri Nets provide the
primitives for modelling concurrency and synchronisation while the
Standard ML programming language provides the primitives for modelling
data and data manipulation. CPNs have been successfully applied for
the modelling and verification of many protocols, including Internet
protocols such as the TCP, DCCP, and DYMO protocols
\cite{billington04,acpn2012}. Formal modelling and verification have
been useful in gaining insight into the operation of the protocols
considered and have resulted in improved protocol
specifications. However, earlier work has not fully leveraged the
investment in modelling by also taking the step to automated code
generation as a way to obtain an implementation of the protocol under
consideration.

% PetriCode approach 

In earlier work~\cite{SKK13-2}, we have proposed the PetriCode
approach and developed a supporting software
tool~\cite{petricodePaper} for automatically generating protocol
implementations based on CPN models.  The basic idea of the approach
is to enforce particular modelling patterns and annotate the CPN
models with code generation \concept{pragmatics}. The pragmatics are
bound to code generation templates and used to direct a template-based
model-to-text transformation that generates the protocol
implementation. As part of earlier work, we have demonstrated the use
of the PetriCode approach on small protocols. In addition, we have
shown that our approach supports code generation for multiple
platforms, and that it leads to code that is readable and also %upwards
%and downwards 
 compatible with other software~\cite{ecmfa14}.

% the web socket protocol and contributions of this work

In the present work we consider the application of our code generation 
approach as
implemented in the PetriCode tool to obtain protocol software
implementing the IETF WebSocket protocol~\cite{fette2011websocket}
protocol for the Groovy language and platform. This demonstrates that
our approach and tool scales to industrial-sized protocols. The
WebSocket protocol is a relatively new protocol %currently under
developed by the IETF. The WebSocket protocol makes it possible to
upgrade an HTTP connection to an efficient message-based full-duplex
connection.  WebSocket has already become a popular protocol for
several web-based applications such as games and media streaming services
where bi-directional communication with low latency is needed.

The contributions of our work include showing how we have been able to
model the WebSocket protocol following the PetriCode modelling
conventions. Furthermore, we perform formal verification of the CPN
model prior to code generation, and test the implementation for
interoperability against the Autobahn WebSocket test-suite
\cite{Autobahn} resulting in 97\% and 99\% success rate for the client
and server implementation, respectively. The tests show that the cause
of test failures were mostly due to local and trivial errors in newly
written code-generation templates, and not related to the overall
logical operation of the protocol as specified by the CPN
model. Finally, we demonstrate in this paper that the generated code
is interoperable with other WebSocket implementations.
  
\paragraph{Acknowledgement.} The results presented in this invited talk is based on joint work with Kent I.F. Simonsen, Bergen University College and the Technical University of Denmark, and Ekkart Kindler, the Technical
University of Denmark.

\bibliographystyle{plain}

%\bibliography{bib}

\documentclass[a4paper]{llncs}

\usepackage[utf8]{inputenc}
\usepackage[T1]{fontenc}
\usepackage{url}
\usepackage{graphicx}
\usepackage{subfigure}
\usepackage{listings}
\usepackage{authblk}
%\usepackage{amsthm}
\usepackage{amssymb}

\usepackage{tabularx}
\usepackage{verbatim}
\usepackage{wrapfig}
\usepackage{capt-of}

\input{macros.tex}
  
%\title{Implementing the WebSocket Protocol \\ based on Coloured Petri Nets Modelling and \\ Automated Code Generation}
 
\title{An Approach for the Engineering of Protocol Software from Coloured Petri Net Models: \\ A Case Study of the IETF WebSocket Protocol} 
\author{Lars Michael Kristensen}
\institute{Department of Computing, Bergen University College, Norway
\\
Email: \email{lmkr@hib.no} 
}

\pagestyle{empty}
\begin{document}
  
\maketitle

\begin{center}
{\textbf{Invited Talk}}
\end{center}

% The relevance of formal modelling of protocols 
The vast majority of software systems today can be characterised as
concurrent and distributed systems as their operation inherently
relies on protocols executed between independently scheduled software
components. The engineering of correct protocols can
be a challenging task due to their complex behaviour which may result
in subtle errors if not carefully designed. Ensuring
interoperability between independently made implementations is also
challenging due to ambiguous protocol specifications.  Model-based
software engineering offers several attractive benefits for the
implementation of protocols, including automated code generation for
different platforms from design-level models. Furthermore, the use of
formal modelling in combination with model checking
provides techniques to support the development of reliable protocol
implementations.

% CPNs and protocol modelling and verification
Coloured Petri Nets (CPNs) \cite{CPNTOOLSPAPER} is formal language
combining Petri Nets with a programming language to obtain a modelling
language that scales to large systems. In CPNs, Petri Nets provide the
primitives for modelling concurrency and synchronisation while the
Standard ML programming language provides the primitives for modelling
data and data manipulation. CPNs have been successfully applied for
the modelling and verification of many protocols, including Internet
protocols such as the TCP, DCCP, and DYMO protocols
\cite{billington04,acpn2012}. Formal modelling and verification have
been useful in gaining insight into the operation of the protocols
considered and have resulted in improved protocol
specifications. However, earlier work has not fully leveraged the
investment in modelling by also taking the step to automated code
generation as a way to obtain an implementation of the protocol under
consideration.

% PetriCode approach 

In earlier work~\cite{SKK13-2}, we have proposed the PetriCode
approach and developed a supporting software
tool~\cite{petricodePaper} for automatically generating protocol
implementations based on CPN models.  The basic idea of the approach
is to enforce particular modelling patterns and annotate the CPN
models with code generation \concept{pragmatics}. The pragmatics are
bound to code generation templates and used to direct a template-based
model-to-text transformation that generates the protocol
implementation. As part of earlier work, we have demonstrated the use
of the PetriCode approach on small protocols. In addition, we have
shown that our approach supports code generation for multiple
platforms, and that it leads to code that is readable and also %upwards
%and downwards 
 compatible with other software~\cite{ecmfa14}.

% the web socket protocol and contributions of this work

In the present work we consider the application of our code generation 
approach as
implemented in the PetriCode tool to obtain protocol software
implementing the IETF WebSocket protocol~\cite{fette2011websocket}
protocol for the Groovy language and platform. This demonstrates that
our approach and tool scales to industrial-sized protocols. The
WebSocket protocol is a relatively new protocol %currently under
developed by the IETF. The WebSocket protocol makes it possible to
upgrade an HTTP connection to an efficient message-based full-duplex
connection.  WebSocket has already become a popular protocol for
several web-based applications such as games and media streaming services
where bi-directional communication with low latency is needed.

The contributions of our work include showing how we have been able to
model the WebSocket protocol following the PetriCode modelling
conventions. Furthermore, we perform formal verification of the CPN
model prior to code generation, and test the implementation for
interoperability against the Autobahn WebSocket test-suite
\cite{Autobahn} resulting in 97\% and 99\% success rate for the client
and server implementation, respectively. The tests show that the cause
of test failures were mostly due to local and trivial errors in newly
written code-generation templates, and not related to the overall
logical operation of the protocol as specified by the CPN
model. Finally, we demonstrate in this paper that the generated code
is interoperable with other WebSocket implementations.
  
\paragraph{Acknowledgement.} The results presented in this invited talk is based on joint work with Kent I.F. Simonsen, Bergen University College and the Technical University of Denmark, and Ekkart Kindler, the Technical
University of Denmark.

\bibliographystyle{plain}

%\bibliography{bib}

\input{lmk_pnse.bbl}
\end{document}	

\end{document}	

\end{document}	

\end{document}	
